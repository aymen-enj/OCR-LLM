\documentclass[10pt, aspectratio=169]{beamer}

% --- THEME & COULEURS ---
\usetheme{metropolis}
\metroset{progressbar=frametitle}
\metroset{block=fill}

% Couleurs personnalisées (FST Settat Blue)
\definecolor{myblue}{HTML}{003366}
\definecolor{myred}{HTML}{A30000}
\definecolor{mygrey}{HTML}{F0F0F0}

\setbeamercolor{palette primary}{bg=myblue, fg=white}
\setbeamercolor{frametitle}{bg=myblue, fg=white}
\setbeamercolor{progress bar}{fg=myred, bg=mygrey}
\setbeamercolor{alerted text}{fg=myred}

% --- PACKAGES ---
\usepackage[utf8]{inputenc}
\usepackage[french]{babel}
\usepackage{graphicx}
\usepackage{booktabs}
\usepackage{tikz}
\usepackage{hyperref}

% --- INFO PROJET ---
\title{Extraction et Analyse Intelligente de Documents}
\subtitle{Ultimate OCR \& LLM Parser v3.3}
\author{Aymen Ennaji}
\institute{FST Settat}
\date{} % Date supprimée

% --- CUSTOM TITLE PAGE ---
% --- CUSTOM TITLE PAGE ---
% --- CUSTOM TITLE PAGE ---
\setbeamertemplate{title page}{
    \begin{center}
        % 1. On remonte moins fort (ou pas du tout) pour laisser de la place en haut
        \vspace{0.4cm} 
        
        % 2. En-tête avec alignement vertical centré [c]
        \begin{columns}[c]
            
            % --- IMAGE GAUCHE ---
            \column{0.20\textwidth} % Un peu plus de place pour l'image
            \centering
            % IMPORTANT : On fixe la HAUTEUR (height) pour s'aligner sur le texte
            \includegraphics[height=1.3cm, keepaspectratio]{609-6094376_uh1.png}
            
            % --- TEXTE CENTRAL ---
            \column{0.60\textwidth}
            \centering
            {\scriptsize \textbf{ROYAUME DU MAROC} \\ Université Hassan 1\textsuperscript{er} \\ Faculté des Sciences et Techniques de Settat}
            
            % --- IMAGE DROITE ---
            \column{0.20\textwidth}
            \centering
            \includegraphics[height=1.5cm, keepaspectratio]{logo-fst.png}
            
        \end{columns}
        
        \vspace{0.2cm}
        \hrule height 0.5pt
        \vspace{0.3cm}
        
        % ... LE RESTE NE CHANGE PAS ...
        
        % CONTEXTE ACADÉMIQUE
        {\scriptsize \textsc{Filière : LST Génie Informatique (LST GI)}} \\
        \vspace{0.1cm}
        {\scriptsize \textsc{Module : Architecture des Ordinateurs}}
        
        \vspace{0.3cm}
        
        % TITRE DU PROJET
        \begin{beamercolorbox}[sep=6pt,center]{title}
            \usebeamerfont{title} \Large \textbf{Extraction et Analyse} \\ 
            \vspace{0.1cm}
            \Large \textbf{Intelligente de Documents}
        \end{beamercolorbox}
        
        \vspace{0.1cm}
        {\usebeamerfont{subtitle} \small \textit{Ultimate OCR \& LLM Parser}}
        
        \vspace{0.1cm} % Un peu plus d'espace avant les noms
        
        % AUTEURS ET ENCADRANT
        \begin{columns}[t]
            \column{0.5\textwidth}
            \centering
            \textbf{\scriptsize \textsc{Réalisé par :}} \\
            \vspace{0.1cm}
            {\small ENNAJI Aymen} \\
            {\small ELKETTANI Ahmed}
            
            \column{0.5\textwidth}
            \centering
            \textbf{\scriptsize \textsc{Encadré par :}} \\
            \vspace{0.1cm}
            {\small Pr. Hicham BENALLA}
        \end{columns}
        
        \vspace{0.3cm}
        \hrule height 0.5pt
        \vspace{0.2cm}
        
        % PIED DE PAGE
        {\scriptsize Année Universitaire : \textbf{2025 - 2026}}
    \end{center}
}

\begin{document}

% -----------------------------------------------------------------------------
% SLIDE DE TITRE
% -----------------------------------------------------------------------------
{
\setbeamertemplate{footline}{} 
\begin{frame}
    \titlepage
\end{frame}
}

% -----------------------------------------------------------------------------
% SOMMAIRE
% -----------------------------------------------------------------------------
\begin{frame}{Sommaire}
    \setbeamertemplate{section in toc}[sections numbered]
    \tableofcontents
\end{frame}

% -----------------------------------------------------------------------------
% SECTION 1: INTRODUCTION
% -----------------------------------------------------------------------------
\section{Introduction}

\begin{frame}{Contexte et Problématique}
    \begin{columns}
        \column{0.5\textwidth}
        \textbf{Le Contexte :}
        \begin{itemize}
            \item Augmentation massive des documents numériques (PDF, Images).
            \item Besoin d'automatisation dans les entreprises.
        \end{itemize}
        
        \column{0.5\textwidth}
        \begin{alertblock}{Le Problème}
        Les outils classiques (grep, regex) sont inefficaces sur des documents non structurés (scans, factures variées).
        \end{alertblock}
    \end{columns}

    \vspace{0.5cm}
    \textbf{Notre Objectif :}
    Créer une solution \textbf{hybride} alliant OCR (Vision) et LLM (Intelligence) pour structurer n'importe quel document.
\end{frame}

% -----------------------------------------------------------------------------
% SECTION 2: ARCHITECTURE
% -----------------------------------------------------------------------------
\section{Architecture Système}

\begin{frame}{Technologies Utilisées}
    Une stack technologique moderne et robuste :
    
    \begin{itemize}
        \item \textbf{Langage} : Python 3.11 
        \item \textbf{Moteur OCR} : Tesseract (Google) + Poppler
        \item \textbf{Intelligence Artificielle} : Ollama (Llama 3.2 / Mistral) en local
        \item \textbf{Interface Graphique} : CustomTkinter (Moderne)
    \end{itemize}
\end{frame}

\begin{frame}{Diagramme de Classes (UML)}
    \begin{center}
        \includegraphics[height=0.9\textheight]{diagramme_classes.png}
    \end{center}
\end{frame}

\begin{frame}{Flux de Traitement}
    \begin{columns}
        \column{0.4\textwidth}
        \textbf{Étapes Clés :}
        \begin{enumerate}
            \item \textbf{Input} : Chargement PDF/Img.
            \item \textbf{Smart Ext.} : Décision Native vs OCR.
            \item \textbf{Classif.} : Détection type doc.
            \item \textbf{LLM} : Structuration JSON.
        \end{enumerate}

        \column{0.6\textwidth}
        \includegraphics[width=\textwidth]{diagramme_flux.png}
    \end{columns}
\end{frame}

\begin{frame}{Stratégie : Prétraitement d'Image}
    \begin{columns}
        \column{0.4\textwidth}
        \textbf{Optimisation OCR :}
        \begin{itemize}
            \item Conversion Grayscale.
            \item \textbf{Deskew} : Correction auto de l'inclinaison.
            \item Renforcement du contraste (+60\%).
        \end{itemize}

        \column{0.5\textwidth}
        \includegraphics[width=\textwidth]{schemas_pres/schema_preprocess.png}
    \end{columns}
\end{frame}

\begin{frame}{Stratégie : Logique de Fallback}
    \begin{columns}
        \column{0.4\textwidth}
        \textbf{Approche Hybride :}
        \begin{itemize}
            \item \textbf{Priorité} : Extraction native (Markdown) si possible.
            \item \textbf{Fallback} : Bascule vers OCR Tesseract si scan détecté.
            \item Garantit qualité vs robustesse.
        \end{itemize}

        \column{0.6\textwidth}
        \includegraphics[width=\textwidth]{schemas_pres/schema_fallback.png}
    \end{columns}
\end{frame}

\begin{frame}{Intelligence Artificielle (LLM)}
    \begin{columns}
        \column{0.4\textwidth}
        \textbf{Construction du Prompt :}
        \begin{itemize}
            \item Injection du \textbf{Schéma JSON} cible.
            \item Ajout des règles métier spécifiques.
            \item Combinaison avec le texte brut extrait.
            \item Inférence locale sur Ollama.
        \end{itemize}

        \column{0.6\textwidth}
        \includegraphics[width=\textwidth]{schemas_pres/schema_llm.png}
    \end{columns}
\end{frame}

% -----------------------------------------------------------------------------
% SECTION 3: DESIGN & CONCEPTION
% -----------------------------------------------------------------------------
\section{Conception Détaillée}

\begin{frame}{Diagramme de Séquence}
    Illustration du workflow multi-thread :
    \begin{center}
        \includegraphics[width=0.6\textwidth]{diagramme_sequence.png}
    \end{center}
\end{frame}

\begin{frame}{Design Patterns}
    \begin{block}{Patterns Clés}
    \begin{itemize}
        \item \textbf{Facade} : \texttt{OCRApp} centralise la complexité.
        \item \textbf{Strategy} : Bascule automatique Native $\leftrightarrow$ OCR.
        \item \textbf{Observer} : Callbacks pour la mise à jour de l'UI (Progression).
        \item \textbf{Worker Thread} : UI non bloquante durant le traitement lourd.
    \end{itemize}
    \end{block}
\end{frame}

\begin{frame}{Architecture Logicielle}
    \begin{columns}
        \column{0.4\textwidth}
        \textbf{Structure 3-Tiers :}
        \begin{itemize}
            \item \textbf{Présentation} : GUI CustomTkinter.
            \item \textbf{Domaine} : Orchestrateur OCR et Logique métier.
            \item \textbf{Infra} : Wrappers Tesseract et Ollama.
        \end{itemize}

        \column{0.6\textwidth}
        \includegraphics[width=\textwidth]{schemas_pres/schema_arch.png}
    \end{columns}
\end{frame}

\begin{frame}{Modèle Asynchrone}
    \begin{columns}
        \column{0.4\textwidth}
        \textbf{Non-bloquant :}
        \begin{itemize}
            \item Thread UI principal fluide.
            \item \textbf{Worker Thread} pour tâches lourdes (OCR/IA).
            \item Callbacks \texttt{after()} pour mise à jour UI safe.
        \end{itemize}

        \column{0.6\textwidth}
        \includegraphics[width=\textwidth]{schemas_pres/schema_threading.png}
    \end{columns}
\end{frame}

% -----------------------------------------------------------------------------
% SECTION 4: RÉSULTATS & DÉMO
% -----------------------------------------------------------------------------
\section{Interface & Démonstration}

\begin{frame}{Interface Graphique (GUI)}
    \begin{columns}
        \column{0.4\textwidth}
        \textbf{Fonctionnalités :}
        \begin{itemize}
            \item Drag \& Drop intuitif.
            \item Logs en temps réel.
            \item Export JSON.
        \end{itemize}
        
        \column{0.6\textwidth}
        % Placeholder for a screenshot if user takes one, otherwise descriptive text
        \begin{center}
            \Huge \textbf{LIVE DEMO}
            
            \large Lancement de l'application...
        \end{center}
    \end{columns}
\end{frame}

\begin{frame}{Mécanisme Drag \& Drop}
    \begin{columns}
        \column{0.4\textwidth}
        \textbf{Expérience Utilisateur :}
        \begin{itemize}
            \item Support natif OS (Windows/Mac).
            \item Événement \texttt{<<Drop>>} capturé par TkinterDnD.
            \item Chargement instantané du fichier.
        \end{itemize}

        \column{0.6\textwidth}
        \includegraphics[width=\textwidth]{schemas_pres/schema_dnd.png}
    \end{columns}
\end{frame}

\begin{frame}{Comparatif Performance}
    \begin{table}
        \centering
        \begin{tabular}{lcc}
            \toprule
            \textbf{Critère} & \textbf{Approche Classique} & \textbf{Notre Solution} \\
            \midrule
            Format & Structuré uniquement & Tout (Scan/PDF/Img) \\
            Précision & Faible (OCR brut) & \textbf{Élevée (OCR + LLM Correction)} \\
            Sortie & Texte brut & \textbf{JSON Structuré} \\
            Confidentialité & Cloud (Souvent) & \textbf{100\% Local} \\
            \bottomrule
        \end{tabular}
    \end{table}
\end{frame}

\begin{frame}{Schémas de Sortie : CV}
    \begin{columns}
        \column{0.4\textwidth}
        \textbf{Extraction Candidat :}
        \begin{itemize}
            \item Identité et Contacts.
            \item Compétences techniques.
            \item Expériences et Formations structurées.
        \end{itemize}

        \column{0.4\textwidth}
        \includegraphics[width=\textwidth]{screenshots/code_schema_cv.png}
    \end{columns}
\end{frame}

\begin{frame}{Schémas de Sortie : Facture}
    \begin{columns}
        \column{0.4\textwidth}
        \textbf{Extraction Comptable :}
        \begin{itemize}
            \item Émetteur / Client.
            \item Lignes d'articles (Tableaux).
            \item Totaux HT, TVA, TTC.
        \end{itemize}

        \column{0.6\textwidth}
        \includegraphics[width=\textwidth]{screenshots/code_schema_facture.png}
    \end{columns}
\end{frame}

% -----------------------------------------------------------------------------
% SECTION 5: CONCLUSION
% -----------------------------------------------------------------------------
\section{Conclusion}

\begin{frame}{Bilan et Perspectives}
    \begin{columns}
        \column{0.5\textwidth}
        \textbf{Réalisations $\checkmark$}
        \begin{itemize}
            \item Pipeline complet OCR + LLM.
            \item Application GUI moderne et packaging .exe.
            \item Précision > 95\% sur les tests.
        \end{itemize}
        
        \column{0.5\textwidth}
        \textbf{Perspectives $\rightarrow$}
        \begin{itemize}
            \item Traitement par lot (Batch).
            \item Support de Tableaux complexes.
            \item API REST (FastAPI).
        \end{itemize}
    \end{columns}
    
    \vspace{1cm}
    \centering
    \Large \textbf{Merci de votre attention !}
\end{frame}

\end{document}
